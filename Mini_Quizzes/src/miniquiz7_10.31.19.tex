\documentclass[12pt, notitlepage]{article}   	% use "amsart" instead of "article" for AMSLaTeX format
\usepackage{geometry}                		% See geometry.pdf to learn the layout options. There are lots.
\geometry{a4paper}                   		% ... or a4paper or a5paper or ... 
%\geometry{landscape}                		% Activate for rotated page geometry
\usepackage[parfill]{parskip}    		% Activate to begin paragraphs with an empty line rather than an indent
\usepackage{graphicx}				% Use pdf, png, jpg, or eps§ with pdflatex; use eps in DVI mode
								% TeX will automatically convert eps --> pdf in pdflatex

\usepackage{hyperref}

%Use for images
\usepackage{graphicx}
\graphicspath{ {./images/} }

%SetFonts

\usepackage[T1]{fontenc}
\usepackage[utf8]{inputenc}

\usepackage{tgbonum}

%SetFonts

\title{
	\textbf{
		Mini-Quiz 7
	} \\
	\large BIOL 4301/6301 \\
	\large October 31, 2019 \\
}

\date{\vspace{-5ex}}

\def\wl{\par \vspace{\baselineskip}}

\begin{document}

{\fontfamily{phv}\selectfont %select helvetica (code = phv)

\large{Name:}

{\let\newpage\relax\maketitle}

\section{\small{A primary issue in all of science is the "one Earth" problem.
This states that scientists cannot fully predict the impact of anthropogenic activity
on ecosystem services because we do not have a second Earth to experiment with.
Nonetheless, we need these predictions in order to best prepare humanity for the future.
\\
\\
Pick an ecosystem service and a time and spatial scale (e.g., something important to you). 
Design a study that will allow you to predict how this
service will change in the future at the scale you chose. Assume you have unlimited resources. 
Note the benefits and any limitations to your design.}}


} %end font selection

\end{document}