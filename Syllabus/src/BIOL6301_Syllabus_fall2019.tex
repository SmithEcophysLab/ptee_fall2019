\documentclass[12pt, notitlepage]{article}   	% use "amsart" instead of "article" for AMSLaTeX format
\usepackage{geometry}                		% See geometry.pdf to learn the layout options. There are lots.
\geometry{a4paper}                   		% ... or a4paper or a5paper or ... 
%\geometry{landscape}                		% Activate for rotated page geometry
\usepackage[parfill]{parskip}    		% Activate to begin paragraphs with an empty line rather than an indent
\usepackage{graphicx}				% Use pdf, png, jpg, or eps§ with pdflatex; use eps in DVI mode
								% TeX will automatically convert eps --> pdf in pdflatex

\usepackage{hyperref}
		
%SetFonts

\usepackage[T1]{fontenc}
\usepackage[utf8]{inputenc}

\usepackage{tgbonum}

%SetFonts

\title{
	\textbf{
		BIOL 6301-029
	} \\
	\large Principles of Terrestrial Ecosystem Ecology \\
	\large Fall 2019
}

\date{\vspace{-5ex}}

\begin{document}

{\fontfamily{phv}\selectfont %select helvetica (code = phv)

\maketitle

\section{Course Description}
Students in this course will learn the fundamentals of ecosystem ecology.
This will include interactions between biological organisms and themselves as well as
their environment. Concepts taught will include water and energy cycling as well as carbon
and nutrient flows in both natural and managed systems.
This will include aboveground and belowground processes.
As ecosystem ecology is the largest scale of ecology, the class will also cover necessary
concepts of individual, population, and community ecology.
Spatial extent of processes will extend to the globe. Temporal extent will extend to 
millenia. The class will consider applied aspects of global change and human decision making.
Students will be evaluated on their 
ability to discuss and disseminate ecosystem ecology topics.

\subsection{Class Time and Location}
Tuesdays and Thursdays 12:30-13:50

Biology Building Room 102/103

\subsection{Instructor}
Dr. Nick Smith \par
Biology Building Rm. 215 \par
806-834-7363 \par
nick.smith@ttu.edu \par
\textit{Meetings by appointment}

\subsection{Text}
Principles of Terrestrial Ecosystem Ecology (2nd Edition; 2011) 
by Chapin, Matson, and Vitousek \par
The book can be accessed from Springer here: 
\url{https://www.springer.com/us/book/9780387783406}. Click on "Access this title on 
SpringerLink." It can also be accessed through the TTU library.

\section{Course Materials}
All course materials, including lecture slides, readings, activities, and code 
will be posted to a GitHub repository for the course.
The primary repository address is
\url{https://github.com/SmithEcophysLab/ptee_fall2019}.
The repository will include the syllabus, daily class notes, readings, and mini-quizzes.
The repository will also include other miscellaneous class materials as the semester
progresses. A README file will contain information on the repository, including
links to different sections at 
\url{https://github.com/SmithEcophysLab/ptee_fall2019/README.md}.

\section{Learning Objective}
This course will broadly focus on understanding the interactions between biological
organisms and their environment that drive cycles of energy, water, carbon, and nutrients
at local to global scales.
An emphasis will be placed on how these processes influence humanity in a changing world.
Applied concepts will consider human decisions as a means to reduce the negative impact
of global change. Class activities will be based on discussion and dissemination of ideas, 
including classic and recent scientific literature. 
Topics will be flexible and modified to match student interests where possible.

\section{Attendance Policy}
Attendance will not be taken, but is strongly recommended. 
In class activity points will only be granted if students are in class.
Makeups will not be granted.

\section{Course Assessment}
\subsection{\textit{Participation and Engagement}}
Being an active and engaged participant in the class will benefit your understanding
of material as well as your peers'. Examples include asking questions, providing feedback,
and facilitating discussion.

\subsection{\textit{Mini-quizzes}}
Short “quizzes” will be given in class each week (typically on Thursdays). 
These quizzes will be used to stimulate discussion and to assess how well 
prior concepts were understood by the class. 
These will be graded for completion and participation in the ensuing class discussion.

\subsection{\textit{Reading feedback}}
Each week students will be required to read a a section of the book 
and produce a short summary as well as two questions that arose during their 
reading. 

\subsection{\textit{In-class discussion lead and feedback}}
Each student will be required to lead one Thursday discussion on recent literature. 
This will involve choosing an article for the class to read, presenting the article, 
and leading a discussion related to the article. Students must read some of the cited
literature integral to the study 
in order to answer relevant questions brought forth during the discussion.
Students not leading the current week’s 
discussion will be required to produce a summary and 
develop two questions based on each week’s article.
Discussion leads will be done in groups of 1-2 students.

\subsection{\textit{Deliverable}}
The primary semester project will be to produce a deliverable.
Broadly, the deliverable should answer a question or solve a problem related 
to ecosystem ecology.
This deliverable could be a literature review, a research project, a podcast, 
an outreach project, etc. Students will first present their idea for their deliverable 
to the class. The class will provide feedback. Students will then produce and present 
their deliverable to the class at the end of the semester. This project must be done 
individually. Students are encouraged to receive help and guidance 
from the instructor as well as the class at large. 

The deliverable will be assessed for completeness, breadth, originality, and presentation.
Students must have their project OKed by the instructor before beginning. Field, lab,
or greenhouse space may be used, subject to availability.

\section{Grading}
Participation and Engagement: 15\% \par
Mini-quizzes: 10\% \par
Reading feedback: 5\% \par
Recent literature lead: 15\% \par
Recent literature feedback: 5\% \par
Deliverable idea proposal: 10\% \par
Deliverable idea feedback: 5\% \par
Final deliverable presentation: 10\% \par
Final deliverable: 25\% \par

Grades will be made available on Blackboard. 
All grades posted at the end of the course will be final, 
unless an error has been made in their calculation.
Please contact Dr. Smith if you feel your grade has been calculated incorrectly.

\section{Grading Scale}
A: $\geq$ 90\% \par
B: 80 – 90\% \par
C: 70 – 80\% \par
D: 60 – 70\% \par
F: $\leq$ 59.9\% \par

\section{Missing In-class Activities}
Students will be required to be in class to receive in-class activity points. 
Please contact Dr. Smith if you plan to miss class for a university function 
\textit{prior to class}. If class is missed due to an illness, 
please let Dr. Smith know as soon as possible and provide Dr. Smith with 
a signed doctor’s note by the start of the next class period.

\section{Special Considerations}
\subsection{Disabling Condition}
Any student who, because of a disability, may require special arrangements in order to 
Any student who, because of a disability, may require special arrangements in order to 
meet the course requirements should contact Dr. Smith as soon as possible to make 
any necessary arrangements. Students should present appropriate verification from Student 
Disability Services. Please note instructors are not 
allowed to provide classroom accommodations to a student until appropriate verification 
from Student Disability Services has been provided. For additional information, you may 
contact the Student Disability Services office at 335 West Hall or 806-742-2405.

\subsection{Religious Holy Days}
"Religious holy day" means a holy day observed by a religion whose places of worship are 
exempt from property taxation under Texas Tax Code §11.20.
A student who intends to observe a religious holy day should make that intention known 
in writing to the instructor prior to the absence. A student who is absent from classes 
for the observance of a religious holy day shall be allowed to take an examination or 
complete an assignment scheduled for that day within a reasonable time after the absence.
A student who is excused may not be penalized for the absence; however, the instructor 
may respond appropriately if the student fails to complete the assignment satisfactorily.

\section{Academic Integrity}
As stated in the Texas Tech University catalog, “The attempt of any students to present 
as their own work that they have not honestly performed is regarded by the faculty and 
administration as a serious offense and renders the offenses liable to serious 
consequences, possibly suspension.” This statement applies to cheating in whatever 
manner, including plagiarism.

\section{TTU Resources for Discrimination, Harassment, and Sexual Violence}
Texas Tech University is committed to providing and strengthening an educational, 
working, and living environment where students, faculty, staff, and visitors are 
free from gender and/or sex discrimination of any kind. Sexual assault, discrimination, 
harassment, and other Title IX violations are not tolerated by the University. 
Report any incidents to the Office for Student Rights and Resolution, 
(806)-742-SAFE (7233) or file a report online at 
\url{titleix.ttu.edu/students}. 

Faculty and staff members at TTU are committed to connecting you to resources on campus. 
Some of these available resources are: 
\begin{itemize}
	\item{TTU Student Counseling Center, 806-742-3674, \url{https://www.depts.ttu.edu/scc}. 
		Provides confidential support on campus.} 
	\item{TTU 24-hour Crisis Helpline, 806-742-5555. 
		Assists students who are experiencing a mental health or interpersonal violence 
		crisis. If you call the helpline, you will speak with a mental health counselor.} 
	\item{Voice of Hope Lubbock Rape Crisis Center, 806-763-7273, 
		\url{https://voiceofhopelubbock.org}.
		24-hour hotline that provides support for survivors of sexual violence.} 
	\item{The Risk, Intervention, Safety and Education (RISE) Office, 806-742-2110, 
		\url{https://www.depts.ttu.edu/rise/}. Provides a range of resources and support 
		options focused on prevention education and student wellness.} 
	\item{Texas Tech Police Department, 806-742-3931, 
		\url{http://www.depts.ttu.edu/ttpd/}. 
		To report criminal activity that occurs on or near Texas Tech campus.}
\end{itemize}

\section{LGBTQIA}
I identify as an ally to the lesbian, gay, bisexual, transgender, queer, intersex, 
and asexual (LGBTQIA) community, and I am available to listen and support you in an 
affirming manner. I can assist in connecting you with resources on campus to address 
problems you may face pertaining to sexual orientation and/or gender identity that could 
interfere with your success at Texas Tech. Please note that additional resources are 
available through the Office of LGBTQIA within the Center for Campus Life, 
Student Union Building Room 201, 
\url{www.lgbtqia.ttu.edu}, 806.742.5433.

\newpage

\section*{Schedule of Topics by Week}
Note: Book chapters in parentheses
08/26/19 – Introductions, semester planning, and the Ecosystem Concept (Ch. 1) \par
09/02/19 – Climate and Soils (Ch. 2, 3) \par
09/09/19 – Water and Energy Balance (Ch. 4) \par
09/16/19 – Carbon Inputs (Ch. 5, 6)
09/23/19 – Carbon Outputs (Ch. 6, 7) \par
09/30/19 – No classes; \textbf{work on your deliverable idea} \par
10/07/19 – \textbf{Deliverable idea proposal presentation} \par
10/14/19 – Plant Nutrient Use and Nutrient Cycling (Ch. 8, 9) \par
10/21/19 – Species and Trophic Dynamics (Ch. 10, 11) \par
10/28/19 – Spatio-temporal heterogeneity (Ch. 12, 13)
11/04/19 – Changes in the Earth System (Ch. 14) \par
11/11/19 – \textbf{Deliverable presentations} \par
11/18/19 – \textbf{Deliverable presentations} \par
11/25/19 – No classes - Happy Thanksgiving! \par
12/02/19 – \textbf{Final deliverables due} \par

\section*{General Weekly Schedule}
Generally, each Tuesday will consist of a lecture by Dr. Smith followed by a discussion
of the reading. Students will turn in their reading
feedback at the end of Tuesday's lecture. Thursdays will generally begin with an in-class
mini-quiz and discussion. 
This will be followed by a discussion of a recent literature article and
(time permitting) an in-class activity.

} %end font selection

\end{document} 
